\documentclass[a4paper]{article}\twocolumn
\setlength{\columnsep}{6mm}

%% Language and font encodings
\usepackage[english]{babel}
\usepackage[utf8x]{inputenc}
\usepackage[T1]{fontenc}

%% Sets page size and margins
\usepackage[a4paper,top=3cm,bottom=2cm,left=2.5cm,right=2.5cm,marginparwidth=1.75cm]{geometry}

%% Useful packages
\usepackage{amsmath}
\usepackage{graphicx}
\usepackage[colorinlistoftodos]{todonotes}
\usepackage[colorlinks=true, allcolors=blue]{hyperref}
\usepackage{url}
\usepackage{apacite}
\AtBeginDocument{\urlstyle{APACsame}}
\usepackage{placeins}
\usepackage{verbatim}

\title{\textbf{Israel and the Syrian Civil War}\\A Security Analysis}
\author{\textit{Sebastian Krantz}}

\begin{document}


\maketitle

%\begin{abstract}
%Your abstract.
%\end{abstract}

\section{Introduction}
Israel and Syria have been antagonistic towards one another since the very foundation of Israel in 1948. From the perspective of Israel one could even contend that Syria has over decades been its most bitter \hspace{1.4mm} enemy \hspace{1.4mm}  among \hspace{1.4mm} its \hspace{1.4mm} direct \hspace{1.4mm} neighbors \newline \cite{Rabinovich2012}. The relationship between the two countries has been characterized by three wars: the 1948 Arab Israeli War, the 1967 Six-Day War and the Yom Kippur War in 1973. During the 1967 war, Israel captured the Golan Heights, which was followed by fierce efforts from Syria to regain them. The Golan Heights have been the greatest subject in their relationship from then on \cite{Rabinovich2012}. Furthermore, both countries have been involved in the Lebanese Civil War (1975 to 1990) and the Lebanon War in 1982. When Bashar al-Assad came to occupy office following the death of his father Hafiz al-Assad in mid 2000, The relationship between the two countries, which had been moving towards a peace agreement in the previous years, cooled down again. Bashar started to strengthen Syrias ties with Iran and Hezbollah, which was followed by a phase of increased arms transfer to Hezbollah (of missiles on particular) in the early 2000's, and the relationship between Israel and Syria had reached a new low point again \cite{Rabinovich2012}. 
The Syrian Civil War opens up a new era of Syria-Israel relationships, and depending of the outcome of the war, relationships between the two countries might improve or deteriorate further. In the meanwhile, the proxy-war raging between pro Assad forces, notably Iran, Hezbollah and Russia, and the Anti-Assad Coalition  as well as groups like ISIS and Al-Nusra is of great relevance to Israel and its national security. This pertains particularly to the leverage of Iran on the Syrian regime, and the influence and military success of Hezbollah in southern Syria and Lebanon \cite{Hanauer2015}. Another important point is whether islamic rebel groups will abide on the border area to Israel or whether they can be drawn from this territory. Though not directly engaged, Israel can thus nevertheless be said to have has several high-level objectives in the Syria War:

\textit{
\begin{enumerate}
 \item To minimize Iranian (and Russian) influence in Syria
 \item To thwart the transfer of (advanced) weapons to Hezbollah and prohibit Syria and/or Iran to pose a military threat to Israel
 \item To promote a weak Assad regime
 \item To subvert Syrian claims to the Golan Heights
 \item To prevent (sunni) rebel groups from establishing military strongholds near the Israeli border
\end{enumerate}
}
\vspace{5mm}
In the long run, Israel would prefer to have a moderate pro-western central government in Syia, with functioning institutions, which is able and willing to resist Iranian interference but yet not strong enough or not motivated to pose a military threat to Israel \cite{Hanauer2015,Rabinovich2012}. At the current state of the conflict achievement of these objectives appear highly unlikely. Officially Israel has chosen to remain impartial in the conflict, which in part may be explained by the unfavorability of any likely outcome of the conflict (either a Iran-friendly Assad, or a country fragmented and controlled by different groups and militias) \cite{Hanauer2015, Rabinovich2012}. At the beginning of the war, the signs were rather disposed towards an Insraeli engagement against Assad, as this would have enabled Israel to refresh its relationship with Turkey and the Arab States, and raised the possibility of a regime change in favor of Israel and against Iran \cite{Rabinovich2012}. But Israel chose to not engage for several reasons which will become clear in the remainder of this text. One of them that is of particular relevance in the early stages of conflict is Assad's public portrayal of the civil uprising in 2011 as a plot directed from the outside, from the US and Israel in particular \cite{Winter2016,greenwood_2013}. At the current state of the conflict, any Israeli engagement in the conflict aside from self-defence is out of reach.  As the conflict evolved though, Iran's deployment of troops, and the strengthening of Hezbollah by the mobilization of thousands of fighters has become a source of unrest for Israel \cite{Hanauer2015}. Acting upon these concerns, Israel has already attacked some convoys or Iranian troops and Hezbollah fighters in a pre-emptive manor, with the implicit consent of Russia \cite{aronson_2017,greenwood_2013}. In doing so, Israel has tried to manover in a position of formal disengagement, while monitoring the conflict very closely, acting on threats and intrusions into its territory, and trying to uphold cooperation with Russia \cite{Hanauer2015}.

\section{Challenges \& Opportunities}%---------------------------------------------------------------------------
Amongst the greatest challenges for Israel at this stage of the War is Iran's high engagement in the conflict, both in terms of support of Hezbollah, but also in terms of troops from the Islamic Revolutionary Guard Corps (IRGC) (a special force tasked with protecting the values of the 1979 islamic revolution and heavily involved in arming and training Hezbollah fighters), The Quds Force (an external operations branch of the IRGC), as well as parts of the regular Iranian army on the ground in Syria \cite{Tabrizi2016}. Iran’s position in Syria has been strengthened by the Russian military intervention on behalf of Assad, which, in targeting sunni militias and the free syrian army, gave Iran and its partners greater room to maneuver and limited Israel’s ability to act militarily \cite{Hanauer2015,ravid_2017,aronson_2017}. In fact Iranian engagement grew from 2000-3000 IRGC officers early in the war to around 10.000 Iranian operatives in Syria by September 2013 \cite{Tabrizi2016}. This contingent of troops was further complemented by a free trade agreement between Israel and Iran in 2012, the transfer of weapons such as falaq rockets and chlorine bombs to the Assad regime, loans to Syria amounting to 4.6 Billion by 2013, and the Iranian backing of several involved shia militias such as Katai’b Al-Imam
Ali \cite{Tabrizi2016}. The remainder of this section will briefly adress the challenges and opportunities connected to each of the 5 geostrategic interests of Israel in the Syrai war identified above. 

\subsection{Containing Iranian Influence and Weapons Transfers to Hezbollah}
Of its many enemies, Iran is certainly the one that poses the most credible existential threat to Israel, especially if it were to acquire nuclear weapons \newline \cite{Hanauer2015}. The pledge to eradicate Israel has been repeated several times by multiple high-level Iranian officials (like 2015 by General Mohammad Ali Jafari commander of the Iranian Revolutionary Guard Corps, Ayatollah Ali Khamenei (2001) and others)  \cite{goldberg_2015,reuters_2015}. Israel’s ambassador to the United States, Michael Oren, stated in 2013 that “the greatest danger to Israel is by the strategic arc that extends from Tehran, to Damascus to Beirut. And we saw the Assad regime as the keystone in that arc” \cite{williams_2013}. Without engaging explicitly in the conflict, Israels ability to contain Iranian operations in Syria is limited to attacking Iranian convoys to Hezbollah and to secure its borders in the Golan Heights \cite{Hanauer2015}. A frightening prospect is that Iran (whose military personell on the ground has risen to 10.000 following September 2013), could concentrate its troops on the Israeli border during, or in the aftermath of, the conflict \cite{Hanauer2015,Tabrizi2016}. Iranian troops and Hezbollah fighters have already conducted operations in southern Syria and established a moderate military infrastructure, enabling them to deploy fighters close to the Israeli border \cite{Hanauer2015}. \newline

Iran has also supplied Syrian forces with chlorine-bombs on October-December 2013 and Falaq-1 and Falaq-2 rocket systems \cite{Tabrizi2016}. These weapons could be directed against Israel and pose a threat to Israels security. In his speech to the United Nations General Assembly in October 2015, Netanyahu claimed that 
"Iran’s proxy Hezbollah smuggled into Lebanon SA-22 mis- siles to down our planes, and [Russian-made] Yakhont cruise missiles to sink our ships. Iran supplied Hezbollah with precision-guided surface-to-surface missiles and attack drones so it can accurately hit any target in Israel" \cite{nethanyahu2015}.
A further unfortunate development for Israel is the current Russian engagement, which in its eyes does little more but facilitate unbounded action of Iran and Hezbollah within Syria \cite{aronson_2017,Rabinovich2012,Hanauer2015}.
These concerns have been exacerbated by Russia’s recent decision to provide Iran with S-300 air defense systems, which constrains Israels capacity of attacking Iranian supply covoys \cite{Hanauer2015}.\newline

%\todo[inline, color=green!40]{Sources on Russian Air-defence constraining Israeli attacks of Iranian convoys??}
Despite these gloomy prospects, there is also a possibility that the shia-axis may be broken by a sustained fragmentation of Syria with sunni groups controlling critical territory near Syrian borders to Iraq and Lebanon \cite{Hanauer2015}. The fighting has also strained Hezbollah and inflicted many casualties on it, opening up a prospect for the militia to actually emerge weakened from the conflict and potentially cut-off from Iranian backing \cite{Hanauer2015}. Israel has also engaged in close conversation with Russia over their conflicting interests. Netanyahu has met Putin in September 2015, soon after Russia deployed additional forces to Syria, where he sought to obtain from Putin a commitment to prevent the Syrian military from transferring additional Russian weapons to Hezbollah \cite{aronson_2017,Hanauer2015}. Another fruit of the visit was an agreement to “coordinate air, naval, and the electromagnetic arenas” to prevent clashes of Russian and Israeli forces \cite{yaakov2015}. Thus although Russia and Isreal follow very different policies in the region, the communication channels are open and warrant a possibility for future cooperation, and Israels hopes that Moscow might employ its newly gained leverage on Assad to contain anti-Israelian Iranian actions might not be unfounded \cite{Hanauer2015}.
%\todo[inline, color=green!40]{"Israel’s outreach to Russia serves another purpose for the Israeli
%government. At a time of highly strained U.S.–Israeli relations, these bilateral discussions—particularly the fact that they have been held at the head-of-state level—send a signal to Washington that Israel has other partners to which it can turn."}

\subsection{A Weakened Assad Regime}
A senior Israeli intelligence officer stationed in the north of Israel famously commented on the outcome of the war in early 2013: \\\\
“Better the devil we know than the demons we can only imagine if Syria falls into chaos and the extremists from across the Arab world gain a foothold there” \cite{frenkel_boyes_2013,timesofisrael_2013} \newline

Israel prefers the "devil they know", namely Assad, over a potentially much greater devil such as unpredictable and ruthless rebel groups to retain power after the war is over \cite{timesofisrael_2013,Hanauer2015}. This stance is largely motivated by Israels historical experience with Assad as a rather predictable and deterrable neighbor, which out of himself is unlikely to attack Israel, in particular if he emerges substantially weakened from the war \cite{Hanauer2015}. The outcome with a weakened Assad ruling over Syria would ostensibly only leave Israel with the Hezbollah problem which they have learnt to deal with in the past, and is therefore preferable from an Israeli point of view to ISIS, Al-Nusra or Iran itself occupying the border region and establishing military infrastructure there in the aftermath of the conflict. In fact in his talks with Putin Netanyahu has already pointed towards the establishment of an Iranian naval base in Syria as a great security threat to Israel \cite{ravid_2017}.

\subsection{The Golan Heights}
Whereas before 2011 the channels for a peace-deal between Israel and Syria still seemed open, the outbreak of violent protests and subsequent civil war  in 2011 have changed the situation dramatically \cite{Hanauer2015}. A central element of any peace deal regarding the Golan heights throughout different negotiation cycles going back to Henry Kissinger's attempts in the early 1970's, was a guarantee of stability for Israeli territory sides Syria \cite{Rabinovich2012}. The war has contemporaneously thwarted any prospect for this stability and therefore legitimizes Israel's control of the Golan Heights as a buffer zone against instability in Syria \cite{Hanauer2015,Rabinovich2012}. The war will most likely be followed by an extended time frame of heightened instability, thus Israel will probably face little opposition in maintaining and further legitimizing its control of the Golan Heights during the next decade. 

\subsection{Sunni Groups}
While Sunni groups such as ISIS and Al-Nusra do not pose an immediate threat to Israel since they are subsumed in battles with other actors, Israel does have a strategic interest in preventing these groups to establish military strongholds in the area, also because these could destabilize its other neighbors (Lebanon, Jordan and Egypt) \cite{Hanauer2015}. An illustration of this threat is exemplified by the following case: In January 2016, Syrian forces backed by Russian airpower attacked ISIS southwest Syria and forced ISIS fighters to flee across the border into Jordan, where around 2000-2500 Jordanians from the partly quite ISIS-sympathetic population have joined ISIS \cite{magid_2016,schenker_2016,Hanauer2015}. The ability of ISIS and other groups to destabilize countries in the region remains ever-present, and there is little Israel can do about it except sharing intelligence information about the movements of these groups with its neighbors \cite{Hanauer2015}. In the face of these more immediate concerns, Israel is less focused on a potantial direct threat from ISIS to its stability. There is even a possibility that sunni groups could operate to Israels advantage in occupying southern Syria and thereby sustaining a long-term conflict with Hezbollah in Lebanon, preventing either of the groups to focus their forces on Israel \cite{Hanauer2015}. The latter possibility certainly remains speculative, thus Israel is arguably doing its best by continuing to closely monitor the groups movements and if possible to prevent the intrusion of radical sunni militias in its border regions or neighboring countries. A final important challenge to Israel is to prevent that the wests antipathy for ISIS combined with the Iranian nuclear deal give Iran a free pass to act in Syria under the premise of fighting ISIS \cite{Hanauer2015}. As retired Israeli Brig. General Michael Herzog writes:\\\\ "the West must
not give Iran a pass. The war against ISIS should be fought with a long-term view for the future of the region, including a clear eyed view of the threat posed by Iran and its agenda for regional hegemony based on anti-Western values, Shi’a dominance, and nuclear capabilities. The desire to secure a nuclear deal with Iran and join hands in fighting ISIS should not obscure these concerns" \cite{Herzog2015}. 
%\todo[inline, color=green!40]{give a bit more context (russias position, Iran naval abse plan, Turkeys role).}

\section{Feasibility}%---------------------------------------------------------------------------
In light of the present advances, pathetically coined the "end game", the stakes for Israel are mixed \cite{aronson_2017}. With respect to the Golan Heights, it is quite certain that Israels legitimacy will not be challenged for several years, given the projected long term instability of the region in the aftermath of the war. Much less promising are the prospects for a contained Iranian influence in the region. It is likely that Iran, and possibly also Turkey will enjoy an expanded sphere of influence as the situation stabilizes, and although Nethanyahu's talks with Russia appear to have been constructive, there are no signs that any actor currently involved has the motivation or power to prevent an Iran from expanding its influence in the region, which in the worst case could lead to Iranian strongholds in the Golan Heights or the establishment of the feared Iranian naval base in southern Syria \cite{friedman_2016,ravid_2017,nethanyahu2015}. Regarding sunni rebel groups, most notably ISIS and Jabhat al-Nusra, there seems to be a consensus among most actors in disfavor of a long term presence of these groups in the region  \cite{ravid_2017}. None of the main actors are interested in leaving Syria behind as a breeding ground for radical islamists, which makes it likely that these groups will ultimately be eliminated from the region, which in the long-term is probably a favorable development for Israel despite the benefits the country draws from the current instability. At the current state of the conflict it also seems very likely that a weakened Assad regime will resume power in Syria and that thus indeed "the devil we know" will take back his old place without posing a direct threat to Israel, but conversely also with few leverage and incentive to constrain Iranian, Turkish or Russian activities on Syrian territory \cite{timesofisrael_2013,frenkel_boyes_2013}.

\section{Theoretical Perspectives}
From a state security perspective it is clear that Israel is constantly evaluating long term threats to its national security posed by current and potential future developments of the conflict and in the region. This is evident in Netanyahu's vehement preemptive action to inform Puting: "I stressed our strong opposition to Iran or its satellites establishing themselves in Syria. We see that Iran is trying to establish a naval base in Syria. This has serious implications for Israeli security...I think the message was internalized." \cite{ravid_2017}. Israel is overall very concerned about threats to its neighbors, mostly to its "ally's" Egypt and Jordan through the activities of radical groups, and about the formation of antagonistic power constellations in its direct environment, which appear to be strongly motivated by \newline \vspace{3mm}\\ an intrinsic interest of preserving national/border security and stability within the region \cite{friedman_2016}. 
Next to the state security perspective, it is also arguably the case that Israel has been enjoying an enhanced power position within the region, and that structural changes to the power balance in the middle east have positively affected Israel \cite{friedman_2016}. From a (structural) realist perspective it is thus in Israels interest to maintain this stronger power-position gained through the disruptions of the arab-spring and the current instability in Syria, and to prevent major structural changes posed by e.g. a rising Iran or Turkey \cite{friedman_2016}. A further point of concern for Israel that is relevant to explaining its policy from both a structural realist and a state security point of view is that its relationship with the US (its main patron since 1967) has cooled down \cite{friedman_2016}. The prospect of a rising and unpredictable Turkey and Irans declared interest and involvement in the war posit potential long term threats to Israel if either actor becomes dominant in the region \cite{friedman_2016}. Currently Israel is benefiting from the chaos of the war, but the fractionalization also provides great opportunities for Turkey and Iran to establish themselves in the region, which in the face of a weakening relationship with the United States, leaves Israel with a dangerous power vacuum and reveals a long term strategic vulnerability, motivating its current search for other allies \cite{friedman_2016}. This long term strategic view thus may explain not only Israels current hesitation to take a definite position in the war in nescience of its outcome (with regards to Iranian and Turkish leverage in particular), but also its determined efforts to console Russia, which it sees as a potential ally complementing the US \cite{aronson_2017,ravid_2017,friedman_2016}. Finally, an in my opinion important but often disregarded perspective in analyzing this conflict, especially when it comes to the policy of Israel, is the constructivist view. Tha‘er Al-Nashef and Ofir Winter (2016), which I quote here at length, write: \newline \\ "A close look at the Syrian, and sometimes also the wider Arab political discourse, shows the vast gulf between the policy of non-intervention Israel adopted on the Syrian civil war and the imagined roles ascribed to Israel by popular conspiracy theories since the war started. Competing forces have used these theories in order to vilify their opponents, shirk their own responsibility for their failures, and harmonize the dissonances that occurred following the shocking regional developments since late 2010. The political discourse described herein relies on a plot-oriented mindset that is part of the Arab cultural and political heritage that tends to read every Israeli utterance, move, and gesture in absurd conspiratorial contexts" \cite{Winter2016}. \newline \\
It is these widely circulated discourses and political imaginations, from the side of Assad in particular the idea that the uprising against him is actually an Israeli undercover operation, and Israels knowledge thereof, which in part motivate Israels policy in the Syria war \cite{Winter2016}. In particular Israel is interested in a good long term relationship with its neighbors and cooperation with countries and groups involved in the conflict that share its objective to weaken Iranian and salafist jihadist influences in the region \cite{Winter2016}. To honor this objective requires Israel to act in ways that will not give conspiracy narratives momentum, but that work towards breaking the barriers of fear and ignorance rampant among Syrians and others when it comes to Israel \cite{Winter2016}. Steps to achieving this could be to maintain a consistent public position of non-intervention, denouncing war crimes, advocating for a stable democracy in Syria and providing humanitarian aid \cite{Winter2016}. Israel has followed these prescriptions in part by consistently underlining its neutrality, providing aid to druse communities near the Golan Heights border and denouncing some of the war crimes committed. Al-Nashef and Ofir Winter (2016) conclude that "replacing the conspiratorial image of Israel with the image of a friendly state seeking good neighborly relations will serve Israel’s broader long term strategic interests, which go much beyond deterrence of its enemies", and I believe part of its policies towards the Syrian crisis are motivated by such considerations.

\section{Conclusion}
Over time the Syria war has evolved into a supraregional conflict of incredible complexity and its actors are driven by complex sets of geopolitical and ideological incentives. Though officially not a party to the conflict, Israel has high stakes in its outcome, and its policy towards the war is similarly driven by a complex interplay of medium and long term security, geostrategical and ideological objectives, the bulk of which this paper tried to capture and analyze in a few paragraphs. Although the ultimate outcome of the war is still difficult to forecast at the time or writing, it is very probably that Israel will remain in its position of formal disengagement.
Needless to say the only turn of events that could foster in Israeli engagement on sides of the rebels at this point would be a significant military engagement of regime, or regime-friendly troops on Israel \cite{Rabinovich2012}. This however seems unlikely as long as the Battlefield in Syria remains hot. Against the backdrop of a strengthened and influential Hezbollah and Iran emerging from the Syrian conflict, a further Arab-Israeli war (with Iran, Hezbolloh and potentially Turkey as actors) following the Syria crisis seems likely \cite{Rabinovich2012,Hanauer2015,friedman_2016}. In response to the current unfavorable developments, Israel should try to develop methods to increase its leverage on the outcome of the war in discrete and non-military ways \cite{Rabinovich2012}. In addition to the avenues suggested earlier, forging stronger ties with Turkey and renewing its engagement in the palestinian question seem to be forward leading in that respect \cite{Rabinovich2012}. Crucial factors for Israeli policy, that at this point yet remain uncertain, are how the Israeli-US relationships will evolve under the Trump administration, and how Israels relationship with Russia will develop. The future of Israels role and leverage in the region are thus currently inherently unstable, entailing a long-term strategic vulnerability that expresses itself in very cautious policies sides Israel which appear to overcast its contemporary power-position during the chaos of the war and in the aftermath of the arab-spring. 
\FloatBarrier

%-----------------------------------------------------------------------------------------------------------------
\begin{comment}

\section{Some examples to get started}

First you have to upload the image file from your computer using the upload link the project menu. Then use the includegraphics command to include it in your document. Use the figure environment and the caption command to add a number and a caption to your figure. See the code for Figure \ref{fig:frog} in this section for an example.

\begin{figure}
\centering
\includegraphics[width=0.3\textwidth]{frog.jpg}
\caption{\label{fig:frog}This frog was uploaded via the project menu.}
\end{figure}

\subsection{How to add Tables}

Use the table and tabular commands for basic tables --- see Table~\ref{tab:widgets}, for example. 

\begin{table}
\centering
\begin{tabular}{l|r}
Item & Quantity \\\hline
Widgets & 42 \\
Gadgets & 13
\end{tabular}
\caption{\label{tab:widgets}An example table.}
\end{table}

\subsection{How to write Mathematics}

\LaTeX{} is great at typesetting mathematics. Let $X_1, X_2, \ldots, X_n$ be a sequence of independent and identically distributed random variables with $\text{E}[X_i] = \mu$ and $\text{Var}[X_i] = \sigma^2 < \infty$, and let
\[S_n = \frac{X_1 + X_2 + \cdots + X_n}{n}
      = \frac{1}{n}\sum_{i}^{n} X_i\]
denote their mean. Then as $n$ approaches infinity, the random variables $\sqrt{n}(S_n - \mu)$ converge in distribution to a normal $\mathcal{N}(0, \sigma^2)$.


\subsection{How to create Sections and Subsections}

Use section and subsections to organize your document. Simply use the section and subsection buttons in the toolbar to create them, and we'll handle all the formatting and numbering automatically.

\subsection{How to add Lists}

You can make lists with automatic numbering \dots

\begin{enumerate}
\item Like this,
\item and like this.
\end{enumerate}
\dots or bullet points \dots
\begin{itemize}
\item Like this,
\item and like this.
\end{itemize}
\end{comment}

\newpage
%\centering
%\nocite{*}
\bibliographystyle{apacite}
\bibliography{bibliography}
\end{document}